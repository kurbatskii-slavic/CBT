\documentclass[a4paper,12pt,titlepage,final]{article}
% при подготовке финальной версии отчёта смените опцию draft на final

\usepackage[T1,T2A]{fontenc}     % форматы шрифтов
\usepackage[utf8x]{inputenc}     % кодировка символов, используемая в данном файле
\usepackage{tikz}                % для создания иллюстраций
\usepackage{geometry}		     % для настройки размера полей
\usepackage{indentfirst}         % для отступа в первом абзаце секции
\usepackage{float}
\usepackage{listings}			 % для текста Make-файла
\usepackage{amsmath}			 % для формул
\usepackage{amssymb}			 % для символов
\usepackage{diagbox}
\usepackage{float}
\newcommand{\cl}[1]{\mathcal{#1}}
\newcommand{\bb}[1]{\mathbb{#1}}
\geometry{a4paper,left=30mm,top=30mm,bottom=30mm,right=30mm}
\begin{document}
\begin{titlepage}
\centering\noindent
{
\begin{minipage}{0.1\textwidth}

\end{minipage}
\hfill
\begin{minipage}{0.77\textwidth}
\begin{center}
\textbf{МОСКОВСКИЙ ГОСУДАРСТВЕННЫЙ УНИВЕРСИТЕТ}\par
\textbf{имени М.В.Ломоносова}\par
\end{center}
\end{minipage}
\hfill
\begin{minipage}{0.1\textwidth}

\end{minipage}
}\par
{
\textbf{Факультет вычислительной математики и кибернетики}\par
\nointerlineskip
\noindent\makebox[\linewidth]{\rule{\textwidth}{0.4pt}}
}
\vfill
{
\Large{\textbf{Практическое задание по курсу}}\par
\Large{\textbf{«Современные вычислительные технологии»}}\par
}\\
{
\Large{\textbf{Задание №1}}\par
}
{
\Large{\textbf{Отчет}}\par
\Large{\textbf{о выполненном задании}}\par
\Large{студента 303 учебной группы факультета ВМК МГУ}\par
\Large{Курбацкого Вячеслава Константиновича}\par
}
\vfill
{\small Москва\\ \the\year{}}
\end{titlepage}

\section{Постановка задачи}
Требуется численно решить следующую краевую задачу Дирихле:
$$
\begin{cases}
-u''(x) = f(x), \quad x \in (0; 1) \\
u(0) = a, \: u(1) = b
\end{cases}
$$
Для этого на отрезке $(0; 1)$ вводится равномерная сетка $\{x_0, x_1, \ldots, x_N\}$, где $x_i = ih$, $h = 1 / N$ - шаг сетки. Обозначив $y_i = u(x_i)$ получим дискретное уравнение, приближающее уравнение Лапласа:
$$-\frac{y_{i-1} -2y_i + y_{i+1}}{h^2} = f(x_i)$$
Уравнения образуют следующую систему:
$$\frac{1}{h^2}
\begin{bmatrix}
2 & -1 & & & & \\
-1 & 2 & -1 & & & \\
& -1 & 2 & -1 & & \\
& & & \ldots & & \\
& & & -1 & 2 & -1 \\
& & & & -1 & 2 \\
\end{bmatrix} = \begin{bmatrix}
y_1 \\
y_2 \\ 
y_3 \\
\ldots \\
y_{N-2} \\
y_{N-1} \\
\end{bmatrix} = \begin{bmatrix}
f(x_1) + a / h^2 \\
f(x_2) \\ 
f(x_3) \\
\ldots \\
f(x_{N-2}) \\
f(x_{N-1}) + b / h^2 \\
\end{bmatrix} $$
Системы с трёхдиагональными матрицами можно решать методом прогонки.
\section{Метод прогонки}
Система уравнений \( Ax = F \) равносильна соотношению:
\begin{equation}
    A_i x_{i-1} + B_i x_i + C_i x_{i+1} = F_i.
\end{equation}
Здесь $A_i, B_i, C_i$ - элементы нижней, главной и верхней диагоналей соответственно.
Метод прогонки основывается на предположении, что искомые неизвестные связаны рекуррентным соотношением:
\begin{equation}
    x_i = \alpha_{i+1} x_{i+1} + \beta_{i+1}, \quad i = n-1, n-2, \dots, 1.
\end{equation}

Тогда выразим \( x_{i-1} \) и \( x_i \) через \( x_{i+1} \) и подставим в уравнение (1):
\begin{equation}
    (A_i \alpha_i \alpha_{i+1} + B_i \alpha_i + C_i) x_{i+1} + A_i \alpha_i \beta_{i+1} + A_i \beta_i + B_i \beta_{i+1} - F_i = 0,
\end{equation}
где \( F_i \) — правая часть \( i \)-го уравнения. Это соотношение будет выполняться независимо от решения, если потребовать:
\begin{equation}
    \begin{cases}
        A_i \alpha_i \alpha_{i+1} + B_i \alpha_i + C_i = 0, \\
        A_i \alpha_i \beta_{i+1} + A_i \beta_i + B_i \beta_{i+1} - F_i = 0.
    \end{cases}
\end{equation}

Отсюда следует:
\begin{equation}
    \begin{cases}
        \alpha_{i+1} = \frac{-C_i}{A_i \alpha_i + B_i}, \\
        \beta_{i+1} = \frac{F_i - A_i \beta_i}{A_i \alpha_i + B_i}.
    \end{cases}
\end{equation}

Из первого уравнения получим:
\begin{equation}
    \begin{cases}
        \alpha_2 = \frac{-C_1}{B_1}, \\
        \beta_2 = \frac{F_1}{B_1}.
    \end{cases}
\end{equation}

После нахождения прогонных коэффициентов \( \alpha \) и \( \beta \), используя уравнение (2), получим решение системы. При этом:
\begin{equation}
    x_i = \alpha_{i+1} x_{i+1} + \beta_{i+1}, \quad i = n-1, \dots, 1.
\end{equation}
\begin{equation}
    x_n = \frac{F_n - A_n \beta_n}{B_n + A_n \alpha_n}.
\end{equation}
Пользуясь этим методом, решим систему и оценим ошибку при разных $N$.
\section{Численные эксперименты}
На графике ниже показано, что при увеличении числа точек действительно происходит уменьшения ошибки. Эксперимент проводился на точном решении $u(x) = \sin5x + \cos3x$, т.е. $f = -u'' =25\sin5x + 9\cos3x$.
\begin{figure}[H]
    \centering
    \includegraphics[width=\textwidth]{svt.png}
    \caption{Зависимость ошибок в C-норме и L2-норме в зависимости от количества точек $N$}
    \label{fig:svt}
\end{figure}
\section{Выводы}
Поставленная краевая задача Дирихле была численно решена в соответствии с вариантом, предоставляющим точное решение. Для решения был использован метод прогонки для систем с трёхдиагональной матрицей. Для полученного численного решения были оценены нормы ошибок (С и L2-нормы) - отклонения от точного решения в узлах равномерной сетки. Для иллюстрации сходимости были построены графики изменения ошибки при различных $N$.
\end{document}

